%%%%%%%%%%%%%%%%%%%%%%%%%%%%%%%%%%%%%%%%%
% Short Sectioned Assignment
% LaTeX Template
% Version 1.0 (5/5/12)
%
% This template has been downloaded from:
% http://www.LaTeXTemplates.com
%
% Original author:
% Frits Wenneker (http://www.howtotex.com)
%
% License:
% CC BY-NC-SA 3.0 (http://creativecommons.org/licenses/by-nc-sa/3.0/)
%
%%%%%%%%%%%%%%%%%%%%%%%%%%%%%%%%%%%%%%%%%

%----------------------------------------------------------------------------------------
%	PACKAGES AND OTHER DOCUMENT CONFIGURATIONS
%----------------------------------------------------------------------------------------

\documentclass[paper=a4, fontsize=11pt]{scrartcl} % A4 paper and 11pt font size

\usepackage[T1]{fontenc} % Use 8-bit encoding that has 256 glyphs
\usepackage{fourier} % Use the Adobe Utopia font for the document - comment this line to return to the LaTeX default
\usepackage[english]{babel} % English language/hyphenation
\usepackage{amsmath,amsfonts,amsthm} % Math packages

\usepackage{sectsty} % Allows customizing section commands
\usepackage[top=5em]{geometry}
\allsectionsfont{\centering \normalfont\scshape} % Make all sections centered, the default font and small caps

\usepackage{fancyhdr} % Custom headers and footers
\pagestyle{fancyplain} % Makes all pages in the document conform to the custom headers and footers
\fancyhead{} % No page header - if you want one, create it in the same way as the footers below
\fancyfoot[L]{} % Empty left footer
\fancyfoot[C]{} % Empty center footer
\fancyfoot[R]{\thepage} % Page numbering for right footer
\renewcommand{\headrulewidth}{0pt} % Remove header underlines
\renewcommand{\footrulewidth}{0pt} % Remove footer underlines
\setlength{\headheight}{5pt} % Customize the height of the header

\numberwithin{equation}{section} % Number equations within sections (i.e. 1.1, 1.2, 2.1, 2.2 instead of 1, 2, 3, 4)
\numberwithin{figure}{section} % Number figures within sections (i.e. 1.1, 1.2, 2.1, 2.2 instead of 1, 2, 3, 4)
\numberwithin{table}{section} % Number tables within sections (i.e. 1.1, 1.2, 2.1, 2.2 instead of 1, 2, 3, 4)

\setlength\parindent{0pt} % Removes all indentation from paragraphs - comment this line for an assignment with lots of text

\usepackage{mathtools}
\usepackage{amssymb}
\usepackage{gensymb}
\usepackage{chngcntr}
\usepackage{csquotes}
\usepackage{flexisym}

\counterwithout{figure}{section}
%----------------------------------------------------------------------------------------
%	TITLE SECTION
%----------------------------------------------------------------------------------------

\newcommand{\horrule}[1]{\rule{\linewidth}{#1}} % Create horizontal rule command with 1 argument of height

\title{	
\normalfont \normalsize 
\textsc{Utah State University, Computer Science Department} \\ [25pt] % Your university, school and/or department name(s)
\horrule{0.5pt} \\[0.4cm] % Thin top horizontal rule
\huge CS 7910 Computational Complexity\\Assignment 4 \\ % The assignment title
\horrule{2pt} \\[0.5cm] % Thick bottom horizontal rule
}

\author{Gopal Menon} % Your name

\date{\normalsize\today} % Today's date or a custom date

\begin{document}

\maketitle % Print the title

\begin{enumerate}
\item (20 points) In this exercise, we consider the following \textit{Efficient Recruiting Problem}.

Suppose you are helping to organize a summer sports camp, and the following problem comes up. The camp is supposed to have at least one counselor who is skilled at each of the $n$ sports covered by the camp (baseball, volleyball, basketball, and so on). They have received job applications from $m$ potential counselors. For each of the $n$ sports, there is some subset of the $m$ applicants qualified in that sport. The question is: For a given number $k < m$, is it possible to hire at most $k$ of the counselors such that for each of the $n$ sports, at least one hired counselor is qualified in it. We call this the \textit{Efficient Recruiting Problem}.

Prove that the Efficient Recruiting problem is NP-Complete.\\

This problem looks very similar to the Set Cover (SC) problem, which we showed in class to be NP-Complete. The Set Cover problem is that given a universal set $U$, and a set $S$ of subsets of $U$, is it possible to find a set of $k$ elements or less of $S$, such that the union of these elements is equal to the universal set $U$. We can designate an instance of the Set Cover problem to be $(U, S, k)$.

Let the set of $n$ sports be designated by $N$, the set of $m$ councillors by $M$ and let $k\textprime$ be the number of councillors that need to be hired, such that $k\textprime < m$ and we have at least one councillor for a sport. We can designate this instance of \textit{Efficient Recruiting Problem} (ERP) to be $(N, M, k\textprime)$.

Given a certificate for the solution to an ERP problem instance, which will be a set of councillors $S_m$, we can verify in time $O(\left | S_m \right |)$, that is polynomial time that, the solution is correct. This shows that the ERP problem is in NP.

In order to reduce SC to ERP, we can convert every element in the universal set $U$ to an element in the set of sports $N$. We can convert each element in the set $S$ of subsets of $U$ to an element in the set of councillors $M$. Each element in the set of councillors will cover the sports corresponding to the elements of the universal set $U$ covered by an element of set $S$. We can set $k\textprime - 1 = k$. So we reduce the case where a union of at most $k$ elements of $S$ covers the universal set $U$, to the case where a union of at most $k$ sports councillors from the set $M$ covers all the sports in the set of sports $N$. This reduction can clearly be done in polynomial time as it is a on-to-one reduction.

In the case where $k$ subsets $S$ can cover the universal set $U$, correspondingly, a set of $k$ councillors can cover all the sports in the set $N$. Also in the case where $k$ councillors can cover all the sports in the set $N$, $k$ subsets $S$ can cover the universal set $U$. 

The above two paragraphs show that ERP is in NP-Hard. Since we know that it is also in NP, it follows that ERP is NP-Complete.

\end{enumerate}

%----------------------------------------------------------------------------------------

\end{document}